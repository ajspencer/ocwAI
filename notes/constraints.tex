\documentclass[11pt]{article}
\usepackage{amsmath, amssymb, amsthm}
\usepackage[retainorgcmds]{IEEEtrantools}
\usepackage[pdftex]{graphicx}
\usepackage[ampersand]{easylist}
\usepackage{tikz}
\usepackage{circuitikz}
\usetikzlibrary{intersections}

\usepackage{fancyhdr}

%Listings stuff
\usepackage{listings}
\usepackage{lstautogobble}
\usepackage{color}

\definecolor{gray}{rgb}{0.5,0.5,0.5}
\newcommand\tab[1][1cm]{\hspace*{#1}}
\lstset{
basicstyle={\small\ttfamily},
tabsize=3,
numbers=left,
numbersep=5pt,
numberstyle=\tiny\color{gray},
stepnumber=2,
breaklines=true,
boxpos=t
}

%Format stuff
\pagestyle{fancy}
\headheight 35pt

%Header info
\chead{\Large \textbf{Constraints}}
\lhead{}
\rhead{}

\begin{document}
\section{interpreting line drawing}
The simplified model:
\begin{easylist}[enumerate]
& World is presented in general position, can't change type of verticies by moving view
& a world that is trihedral, all verticies are connected by three faces
& Three kinds of lines: concave (-), convex(-), boundary (->), boundary lines hide one face and concave, convex show both faces.
\end{easylist} \hfill \break

Boundary lines have arrows based on which way you'd have to walk along it to see the stuff on the right. \\
In this model, there are 18 ways to arrange a label around a junciton. \\
Ls, forks, Ts and arrows are the type of functions we can have, each with three verticies coming together. \\

In the new model, we are going to add:
\begin{easylist}[enumerate]
& cracks
& shadows
& Nontrihedral verticies
& Light
\end{easylist} \hfill \break

This leads to from 4 to 50+ labels and from 18 to thousands of junctions. \\
To correctly label an object: label each of the verticies, and move backwards and see which of them conflict with our correct label. \\
\section{Search and domain reduction}
Graph coloring: problem, unconstrained local constraints cause problems downstream. \\
Solution: look at local constraints, make sure they can't cause problems downstream. \\
Vocab: \\
\begin{easylist}[enumerate]
& variable v: something that cn have an assignment
& value x: something that can be an assignment
& domain d: bag of values 
& constraint c: limit on variable values
\end{easylist} \hfill \break

\textbf{procedure:}\\
For each depth first search assignment: \\
\tab for each variable  $v_i$ considered \\
\tab \tab for each $x_i$ in $D_i$ \\
\tab \tab \tab for each constraint $c(x_i, x_j)$ where $x_j \in D_j$ \\
\tab \tab \tab \tab if $\exists! x_j \ni c(x_i, x_j)$ satisfied remove $x_i$ from $d_i$ \\ 
\tab \tab \tab \tab if $d_i = \emptyset$ backup\\
\hfill \break
Things we can consider in step 2 \\
\begin{easylist}[itemize]
& nothing, doesn't even produce a right answer
& assignment, takes waayy too long
& check that neihbors are valid along with assignment: 9139 dead ends but correct answer in a minute
& Through D with V reduced to one value: less than a thousand constraints checked, fast 
& propogate checking through variables with reduced domains: 0 seconds, 0 deadends 
& everything, once we color the first state check to see if all other 48 can be colored
\end{easylist} \hfill \break
In general, we want to put the most contrainted problems first. \\
\section{visual object recognition}
For years, vision experts thought vision depended on the following levels: \\
\begin{easylist}[enumerate]
& The lowest description level, brightness values are conveyed explicitily in the image
& The brighness changes in the image are described explicitly in the \textbf{primal sketch}
& The surfaces that are implicit in the primal sketch are described explicitly in the \textbf{two-and-one-half-dimension sketch}
& The volumes that are implciit in the two and one half dimension sketch are described expliitly in the \textbf{volume description}
\end{easylist} \hfill \break
No one could make this model work, and we determined we can do everything at the primal sketch level\\
It was shown that to identify a polyhedra you only need 3 images each of 4 verticies and information of the polyhedra's size. \\
In the \textbf{identification model} problem, we want to match each unkown with a polyhedra. The model contains several images, which display feature points, of each polyhedra. \\
In the simple model, assume that we can only rotate images about the y axsis. The new values in rotation can be found by the equations: \\
\begin{align}
x_\theta &= x cos \theta - z sin \theta\\
y_\theta &= y, \\
z_\theta &= x sin \theta + z cos \theta \\
\end{align} \hfill \break

In the problem of predicting where a third image will be based on two previous x values, we used: 
\begin{align}
x_1 &= x cos \theta_{I_{1}}- z sin \theta_{I_1}
x_2 &= x cos \theta_{I_{2}}- z sin \theta_{I_2}
x_0 &= x cos \theta_{I_{0}}- z sin \theta_{I_0}
\end{align} \hfill \break
We seee that $X_{I_0}$ is a linear combination of the other two Xs. \\
Expanding to the general case, x is a linear combination of 3 other x values + a constant. Math is too complicated to prove \\
The general procedure is as follow:
\begin{easylist}[enumerate]
& Unitil a satisfactory match is made or there are no more models in the model library
&& Find four corresponding points in the obsereved image and in a model's three library images
&& Use he corresponding points to determine the coefficents a and b used to predict the x and y coordinate values of other image points. \\
&& Determine wether a satisfactory match is made by comparing the predicted x and y coordinate values with those actually found in the observed image.
& If a satisfactory match occurs, announce the identity of the unknown, otherwise announce failure. 
\end{easylist} \hfill \break

We can use \textbf{correlation} to help identify things. The simple idea is that we take a test object and run it across the picture and see if we can get a match. \\
This is too general, so we work on intermediate features: take an intermediate feature and run it across and see if we can get a match. \\

\end{document}
