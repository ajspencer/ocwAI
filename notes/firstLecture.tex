\documentclass[11pt]{article}
\usepackage{amsmath, amssymb, amsthm}
\usepackage[retainorgcmds]{IEEEtrantools}
\usepackage[pdftex]{graphicx}
\usepackage[ampersand]{easylist}
\usepackage{tikz}
\usepackage{circuitikz}
\usetikzlibrary{intersections}

\usepackage{fancyhdr}

%Listings stuff
\usepackage{listings}
\usepackage{lstautogobble}
\usepackage{color}

\definecolor{gray}{rgb}{0.5,0.5,0.5}
\lstset{
basicstyle={\small\ttfamily},
tabsize=3,
numbers=left,
numbersep=5pt,
numberstyle=\tiny\color{gray},
stepnumber=2,
breaklines=true,
boxpos=t
}

%Format stuff
\pagestyle{fancy}
\headheight 35pt

%Header info
\chead{\Large \textbf{AI first lecture}}
\lhead{}
\rhead{}

\begin{document}
\section{Goal trees}
Solving the integral $\int \frac{-5x^4}{(1-x^2)^{5/a}} dx $ reduce the problem into one of smaller forms that we can solve with memorization. \\
We use \textbf{Problem reduction} to solve problems such as this, with the following transformations:

\begin{easylist}[itemize]
& Safe transformations
&& $\int -f(x) = -\int f(x)$
&& Pull out the constants
&& Sum of the integrals is the integral of the sum
&& If the degree of numerator is greater than degree of denominator, divide it out
& Heuristic transformations
&& $f(sin(x),cos(x), tan(x), cot(x), sec(x), cos(x)) = g(cos(x), sin(x)) = g_2(tan(x), cosec(x)) = g_3(cotan(x), sec(x))$ 
&& $ \int f(tan(x)) dx = \int \frac{f(y)}{1 + y^2} dy $
&& $1-x^2, x = sin(y), 1+x^2 = x * tan(y)$ 
\end{easylist} \hfill \break

And node: what happens when we split a problem into multiple problems that we must solve. \\
Or node: can be solved one of two different ways, we don't care which is which \\
These nodes are in a tree called a problem reduction tree, goal tree, and or tree \\
To decide between things to integrate in an or: measure the depth of funcitonal composition \\
Procedure for solving:

\begin{easylist}[enumerate]
& apply all safe tranformations
& look in the table, see if we are done
& find a problem using depth of functional composition look at every leaf, apply heurisitc transform
& Go back to step 1
\end{easylist} \hfill \break

Vital knowledge:
\begin{easylist}[enumerate]
& What kind
& How represented
& How much
& How used
& What exactly 
\end{easylist}

\end{document}
